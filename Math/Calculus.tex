\documentclass[12pt]{article}
% preamble.tex
\usepackage{amsmath, amssymb, amsthm}
\usepackage{tcolorbox}
\usepackage{graphicx}
\usepackage{hyperref}
\usepackage{enumitem}
\usepackage{fancyhdr}
\usepackage{geometry}
\geometry{margin=1in}

% Custom environments
\newtcolorbox{blue}{colback=blue!5!white, colframe=blue!75!black}
\newtcolorbox{green}{colback=green!5!white, colframe=green!75!black}
\newtcolorbox{red}{colback=red!5!white, colframe=red!75!black}

\begin{document}
\title{Calculus}
\author{Avid David}
\date{\today}
\maketitle
\tableofcontents
\newpage
\part{Integral Calculus in One Variable}
\section{Indefinite Integrals}
\subsection{Integral of a function}
The integral is the inverse of the derivative operator. Given a function $f(x)$, we can find a function $F(x)$ such that $F'(x) = f(x)$. The integral of a function is not unique, as for each $f(x)$, $F(x) + C$ is also an integral of that function.
\begin{concept}
    Both the derivative and the integral are linear operators, so we have:
    \begin{equation*}
        \int [\alpha f(x) + \beta g(x) ]\dd x = \alpha \int f(x)\dd x + \beta \int g(x) \dd x
    \end{equation*}
\end{concept}

\subsection{Ways to calculate indefinite integrals}
\begin{enumerate}
    \item \textbf{Expansion} 
    \par We use the linearity rule to turn a complicated integral into the sum of many simpler ones, then calculate one by one. 
    \item \textbf{Changing the differential expression ($u$-substitution)}
\par If $\mathlarger{\int f(x) \dd x = F(x) + C}$ then $\mathlarger{\int f(u) \dd x = F(u) + C}$, where $u = u(x)$ is a continuously differentiable function. Then, we can change the integrand $g(x)\dd x$ into:
\begin{equation*}
    g(x) = f(u(x))u'(x) \dd x
\end{equation*}
Then the integral turns into:
\begin{equation*}
    \int g(x) \dd x = \int f(u(x))u'(x) \dd x = \int f(u(x)) \dd u = F(u) + C
\end{equation*}
\begin{insight}
    In the simple case $u = ax+b$, we have $\dd u = a \dd x$. Then if $\mathlarger{\int f(x) \dd x = F(x) + C}$ then:
    \begin{equation*}
        \int f(ax+b) \dd x  = \dfrac{1}{a} F(ax+b) +C
    \end{equation*}
\end{insight}
    \item \textbf{Change of variables}
Consider the integral $\mathlarger{I = \int f(x) \dd x}$, where $f(x)$ is a continuous function. We can change $f(x)$ such that we work with functions with known or easier antiderivatives:
\begin{enumerate}
    \item \textbf{Change of variables type 1:}
    \par Let $x = \varphi(t)$, where $\varphi(t)$ is a monotonic and continuously differentiable function. Then:
    \begin{equation*}
        I = \int f(x) \dd x = \int f[\varphi(t)]\varphi'(t) \dd t
    \end{equation*}
    Denote the antiderivative of $g(t) = f[\varphi(t)]\varphi'(t)$ as $G(t)$ and $h(x)$ as the inverse of $x = \varphi (t)$, we then have:
    \begin{equation*}
        \int g(t) \dd t = G(t) + C \Rightarrow I = G[h(x)] + C
    \end{equation*}

    \item \textbf{Change of variables type 2:}
    \par Let $t = \psi(x)$, where $\psi(x)$ is a continuously differentiable function and we can write $f(x) = g[\psi(x)]\psi ' (x)$. Then:
    \begin{equation*}
         I = \int f(x) \dd x = \int g[\psi(x)]\psi'(x) \dd x
    \end{equation*}
    \par Denote the antiderivative of $g(t)$ as $G(t)$, then:
    \begin{equation*}
        I = G[\psi(x)] + C
    \end{equation*}
    \begin{important}
        Remember to change back to the original variable!
    \end{important}
\end{enumerate}
    \item \textbf{Integration by parts}
    Let $u = u(x)$ and $v = v(x)$ be continuously differentiable functions. We know:
    \begin{equation*}
        \dd (uv) = u \dd v + v \dd u \Rightarrow \int \dd (uv) = \int u \dd v + v \dd u
    \end{equation*}
    Then we have the following formula:
    \begin{equation*}
        \int u \dd v = uv - \int v \dd u
    \end{equation*}
    Consider the integral $I = \mathlarger{\int f(x) \dd x}$ We need to express:
    \begin{equation*}
        f(x) \dd x = [g(x)h(x)]\dd x = g(x) [h(x) \dd x] = u \dd v
    \end{equation*}
    then apply the integration by parts formula to $u = g(x)$, $v = \mathlarger{h(x) \dd x}$
\end{enumerate}




    \subsection{Integral of rational functions}
    \par A rational function is one with the form $f(x) = \dfrac{P(x)}{Q(x)}$, where $P(x)$ and $Q(x)$ are polynomials in $x$. If $\deg P(x) < \deg Q(x)$ then it's called a \emph{true rational function.}
    \par Using polynomial division, we can rewrite any rational function:
    \begin{equation*}
        f(x) = H(x) + \dfrac{r(x)}{Q(x)}
    \end{equation*}
    \par The integral of $H(x)$ can easily be computed. As for the true rational function $\dfrac{r(x)}{Q(x)}$, we will use partial fractions to decompose it into four simpler types of functions. First, using the method of Undetermined Coefficients, we can write $Q(x)$ as the product of linear polynomials and quadratic polynomials with no real roots:
    \begin{equation*}
        Q(x) = \prod_{k = 1}^{n}(x-a_k)^{\alpha_k} \prod_{k=1}^{n}(x^2+p_kx + q_k)^{\beta_k}
    \end{equation*}
    Note that this can always be achieved as per the Fundamental Theorem of Algebra. Then, we just need to compute the following four types of integrals:
    \begin{insight}
        \begin{enumerate}
            \item $\mathlarger{\int \dfrac{A}{x-a} \dd x}$
            \item $\mathlarger{\int \dfrac{A}{(x-a)^k} \dd x} \quad (k \geq 2)$
            \item $\mathlarger{\int \dfrac{Mx+N}{x^2 + px + q}\dd x}$
            \item $\mathlarger{\int \dfrac{Mx+N}{(x^2 + px + q)^m}\dd x \quad (m \geq 2)}$
        \end{enumerate}
    \end{insight}
    \par These integrals cover all cases because any partial fraction decomposition over $\mathbb{R}$ involves repeated linear factors and irreducible quadratic factors, possibly raised to powers. The first two ones are simple:
    \begin{enumerate}
        \item $\mathlarger{\int \dfrac{A}{x-a}\dd x = A\ln|x-a| + C}$
        \item $\mathlarger{\int \dfrac{A}{(x-a)^k}\dd x = A\int (x-a)^{-k} = \dfrac{-A}{(k-1)(x-a)^{k-1}} \quad (k \geq 2)}$
    \end{enumerate}
    The third one is a bit more complicated: we need to use a clever substitution, inspired by completing the square, to bring it back to familiar forms:
    \par Let $a = \sqrt{q-\dfrac{p^2}{4}}$ and $t = x + \dfrac{p}{2}$, the integral becomes:
    \begin{align*}
        \int \dfrac{Mx+N}{x^2 + px + q}\dd x & =  \int \dfrac{Mt + (N - \frac{Mp}{2})}{t^2 + a^2} \dd t \\
        & = M \int \dfrac{t}{t^2 + a^2} \dd t +  \left(N-\frac{Mp}{2} \right) \int \dfrac{1}{t^2 + a^2} \dd t
    \end{align*}
    For the first integral, substitute in $u = t^2$ and we have a logarithm. For the second one, divide both the numerator and the denominator by $a^2$ then substitute $u = \dfrac{t}{a}$, we yield the inverse tangent. Then the integral is simply:
    \begin{equation*}
        \int \dfrac{Mx+N}{x^2 + px + q}\dd x = \dfrac{M}{2}\ln(x^2 + px+ q) + \dfrac{2N - Mp}{\sqrt{4q-p^2}}\arctan \dfrac{2x+p}{\sqrt{4q -p^2}} + C
    \end{equation*}
    \par Before moving on to solve the last integral, let's review a bit of complex numbers:
    \begin{concept}
    Some important (and beautiful) formulas::
        \begin{align*}
        \textbf{Euler's Formula:}\quad & e^{ix} = \cos x + i \sin x \\
        \textbf{De Moivre's Formula:} \quad & (\cos x + i\sin x)^n = \cos nx + i \sin nx
        \end{align*}
    \end{concept}
    We will prove it using ODEs. Let $f(x) = e^x$ and $g(x) = \cos x + i \sin x$. We know:
    \begin{align*}
        f'(x) & = ie^{ix} = if(x) \\
        g'(x) & = -\sin x + i\cos x = ig(x)
    \end{align*}
    \par And also the initial values:
    \begin{equation*}
        f(0) = g(0) = 1
    \end{equation*}
    By uniqueness of solutions to first-order linear ODEs, Euler's formula is proven. By substituting $f(nx) = g(nx)$, De Moivre's formula is proven. Then we have:
    \begin{insight}
    \begin{itemize}
        \item $\cos x = \Re(e^{ix}) = \dfrac{e^{ix} + e^{-ix}}{2}$
        \item $\sin x = \Im(e^{ix}) = \dfrac{e^{ix} + e^{-ix}}{2i}$
    \end{itemize}
    \end{insight}
    From this, we can derive a general formula for powers of the sine and cosine:
    \begin{equation*}
        \cos ^n x =  \left(\dfrac{e^{ix}+e^{-ix}}{2}\right) ^n 
        =  \dfrac{1}{2^n} \sum_{k=0}^{n}\binom{n}{k}e^{i(n-2k)x}
    \end{equation*}
    Now if we group $k$ and $n-k$ in pairs to get $\cos(n-2k)$, we get:
    \begin{equation*}
        \cos ^n x = \dfrac{1}{2^n} \sum_{k=0}^{n} \binom{n}{k} \cos ((n-2k)x)
    \end{equation*}
    Analogously, we also have:
    \begin{equation*}
                \sin ^n x = \dfrac{1}{(2i)^n} \sum_{k=0}^{n} \binom{n}{k} (-1)^k \cos (\dfrac{\pi n}{2}-(n-2k)x)
    \end{equation*}
    \par Now, back to the fourth integral, we first do the same thing: change variables $a = \sqrt{q-\dfrac{p^2}{4}}$ and $t = x + \dfrac{p}{2}$. This yields:
    \begin{align*}
        \int \dfrac{Mx+N}{(x^2 + px + q)^m} \dd t= & \int \dfrac{Mt + (N - \frac{Mp}{2})}{(t^2 + a^2)^m} \dd t \\
        = & M\int \dfrac{t}{(t^2 + a^2)^m} \dd t + \left(N-\dfrac{Mp}{2}\right) \int \dfrac{1}{(t^2 + a^2)^m} \dd t
    \end{align*}
    \par Again, to compute the first integral, we use the substitution $u = t^2$. Then it is:
    \begin{equation*}
        M\int \dfrac{t}{(t^2 + a^2)^m} \dd t = \dfrac{-M}{2(m-1)(t^2 + a^2)^{m-1}} +C
    \end{equation*}
    \par For the second integral however, this time we need a different substitution. Let $t = a\tan z$. Then: $t^2 +a ^2 = a^2 \sec ^2 z$, and $\dd t = a\sec ^2 \dd z$.
    Then the integral turns into:
    \begin{equation*}
        \int \dfrac{1}{(t^2 + a^2)^m} \dd t = a^{2m-1} \int \cos ^{2m-2} z \dd z
    \end{equation*}
    Let $I(z) = \mathlarger{\int \cos ^{2m-2} z \dd z} $and now, we can take advantage of the linearity rule and the formula for powers of cosine we just derived. The integral turns into:
    \begin{align*}
        I(z) = & \int \dfrac{1}{2^{2m-2}} \sum_{k=0}^{2m-2} \binom{2m-2}{k}\cos(2(m-k-1)z) \dd z \\
        = & \dfrac{1}{2^{2m-2}} \sum_{k=0}^{2m-2} \binom{2m-2}{k} \dfrac{1}{2m-2-2k} \sin ( 2(m-k-1)z) +C \\
        = & \dfrac{1}{2^{2m-2}} \left[ \binom{2m-2}{m-1}z + \sum_{k=0, k\neq m-1}^{2m-2} \binom{2m-2}{k} \dfrac{1}{2m-2-2k} \sin ( 2(m-k-1)z)\right] +C
    \end{align*}
    Then the integral, as a whole, evaluates to:

\begin{equation*}
\int \frac{Mx + N}{(x^2 + px + q)^m} \dd x
= \frac{-M}{2(m - 1)(x^2 + px + q)^{m - 1}} + \left(N - \frac{Mp}{2} \right) \left( \sqrt{q - \dfrac{p^2}{4}}\right)^{2m-1}\cdot I(z) + C
\end{equation*}
\subsection{Integral of trigonometric functions}
\begin{enumerate}
    \item \textbf{The general method}
    \par Consider the integral $\mathlarger{\int f(\sin x, \cos x)\dd x}$, where the integrand is a rational function in terms of $\sin x$ and $\cos x$. We can use the "universal trigonometric substitution" $x = \tan \dfrac{t}{2}$. Then:
    \begin{equation*}
        \sin x = \dfrac{2t}{1+t^2}; \quad \cos x = \dfrac{1-t^2}{1+t^2}; \quad \tan x = \dfrac{2t}{1-t^2}; \dd t = \dfrac{2 \dd t}{1 + t^2}
    \end{equation*}
    The integrand turns into a rational function in terms of $t$.
    \item \textbf{Integrals of the form $\mathlarger{\int \sin ^m x \cos ^n x \dd x}$, where $m,n$ are positive integers}
    \begin{itemize}
        \item If $m$ is odd, we let $t = \cos x$
        \item If $n$ is odd, we let $t = \sin x$
        \item If both $m$ and $n$ are even, we use power-reduction formulae:
        \begin{equation*}
            \sin ^2 x = \dfrac{1-\cos 2x}{2}; \quad \quad \cos ^2 x = \dfrac{1+\cos 2x}{2}
        \end{equation*}
        \par Then we'll have a similar integral with the form $\mathlarger{\int \sin ^k 2x \cos ^l 2x \dd x}$
    \end{itemize}
    \item \textbf{Special forms of $\mathlarger{\int f(\sin x, \cos x) \dd x}$}
    \begin{itemize}
        \item Let $t = \cos x$ if $f(-\sin x, \cos x) = - f(\sin x, \cos x)$
        \item Let $t = \sin x$ if $f(\sin x, -\cos x) = - f(\sin x, \cos x)$
        \item Let $t = \tan x$ if $f(-\sin x, -\cos x) = f(\sin x, \cos x)$
    \end{itemize}
\end{enumerate}
\subsection{Integral of irrational expressions}
\par There are two main ways to solve these integrals: using trigonometric substitution, and using the Euler substitution. The trig-sub is very intuitive, but the Euler substitution is also very nice:
\begin{concept}
    Let $t = x + \sqrt{x^2 + a}$ for the integral $\mathlarger{\int f(x, \sqrt{x^2 +a}) \dd x}$. Then: 
    \begin{equation*}
        \dd t = 1 + \dfrac{x} {\sqrt{x^2 + a}} \dd x = \dfrac{x + \sqrt{x^2 +a }} {\sqrt{x^2 +a}} \dd x \Rightarrow \dfrac{\dd t}{t} = \dfrac{\dd x}{\sqrt{x^2 + a}}
    \end{equation*}
\end{concept}
\begin{important}
    \textbf{A special integral:}
    \begin{equation*}
        I   = \int \dfrac{\dd x}{\sqrt[n]{x}-1} 
    \end{equation*}
    \par Let $u = \sqrt[n]{x}$. Then $\dd x = n u^{n-1} \dd u$
    \begin{align*}
        I &= n \int \dfrac{u^{n-1}}{u-1}\dd u \\
        &= n \int \dfrac{u^{n-1} - 1}{u-1} + \dfrac{1}{u-1} \dd u \\
        &= n \int \sum_{k=0}^{n-2}u^k + \dfrac{1}{u-1}\dd u \\
        & = n\left(\sum_{k=1}^{n-1} \dfrac{u^k}{k} + \ln{|u-1|} \right) + C \\
        & = n\left(\sum_{k=1}^{n-1} \dfrac{\sqrt[n]{x^k}}{k} + \ln{|\sqrt[n]{x}-1|} \right) + C
    \end{align*}
\end{important}
\section{Definite Integrals}
\subsection{Definition}
Say $f(x)$ is defined and bounded on $[a,b]$. Partition $[a,b]$ into $n$ subintervals $[x_i, x_{i+1}]$ where $a = x_0 < x_1 < \ldots < x_n = b$. In each interval $[x_i,x_{i+1}]$, we choose a point $\xi \in [x_i, x_{i+1}]$ and form the expression
\begin{equation*}
    S_n = \sum_{i=0}^{n-1}f(\xi_i) \Delta x_i
\end{equation*}
where $\Delta x_i = x_{i+1} - x_i$. Here, $S_n$ is the Riemann sum. Denote $\lambda = \mathlarger{\max_{1 \leq i \leq n}} \Delta x_i$. If there exists the limit $I = \mathlarger{\lim_{\lambda \rightarrow 0} S_n}$ that doesn't depend on how we partition $[a,b]$ and how we choose $\xi_i$ then $I$ is called the definite integral of the function $f(x)$ on $[a,b]$, denoted $\mathlarger{\int_a ^bf(x) \dd x}$. Then we say $f(x)$ is integrable on $[a,b]$.
\par We then have defined the definite integral for all $a<b$. We can then define, if $b<a$, $\mathlarger{\int_a ^bf(x) \dd x = - \int_b ^af(x) \dd x}$ and when $a = b$, $\mathlarger{\int_a ^bf(x) \dd x = 0}$.
\subsection{Riemann Integrability}
The sufficient and necessary condition for a bounded function $f(x)$ to be integrable on $[a,b]$ is $\mathlarger{\lim_{\lambda \rightarrow0}(S-s)=0}$, where:
\begin{align*}
    S = & \sum_{i=1}^{n+1} M_i \Delta x_i \quad \quad \quad \quad \quad \hspace{10pt} s = \sum_{i=1}^{n+1} m_i \Delta x_i \\
    M_i = & \sup_{x \in [x_i, x_{i+1}]} f(x) \quad \quad \quad \quad m_i =  \inf_{x \in [x_i, x_{i+1}]} f(x)
\end{align*}
From there, we have a few crucial theorems:
\begin{concept}
    \begin{itemize}
        \item If $f(x)$ is continuous on $[a,b]$ then it is integrable on $[a,b]$.
        \item If $f(x)$ is bounded on $[a,b]$ and has discontinuities on $[a,b]$ then it is integrable on $[a,b]$
        \item If $f(x)$ is bounded and monotonic on $[a,b]$ then it is integrable on $[a,b]$
    \end{itemize}
\end{concept}
\subsection{Properties of the Definite Integral}
\begin{enumerate}
    \item \textbf{Property 1 (Linearity:)}
    \begin{equation*}
        \int_a^b [\alpha f(x) + \beta g(x)]\dd x = \alpha \int_a^b f(x) \dd x + \beta \int_a^b g(x) \dd x
    \end{equation*}
    \item \textbf{Property 2:} Given three closed intervals $[a,b],[b,c],[a,c]$, if $f(x)$ is integrable on the longest interval then it is also integrable on the other two integrals, and:
    \begin{equation*}
        \int_a^b f(x) \dd x = \int_a^c f(x) \dd x + \int_c^b f(x) \dd x
    \end{equation*}
    \item \textbf{Property 3:} Suppose $a<b$. Then:
    \begin{enumerate}
        \item If $f(x) \geq 0 \ \forall \ x \in [a,b]$ then $\mathlarger{\int_a^b f(x) \dd x \geq 0}$
        \item If $f(x) \geq g(x) \ \forall \ x \in [a,b]$ then $\mathlarger{\int_a^b f(x) \dd x \geq \int_a^b g(x) \dd x}$
        \item If $f(x)$ is integrable on $[a,b]$ then $|f(x)|$ is integrable on $[a,b]$ and:
        \begin{equation*}
            \left| \int_a^b f(x) \dd x \right| \leq \int_a^b |f(x)| \dd x
        \end{equation*}
        \item If $m \leq f(x) \leq M \ \forall \ x \in [a,b]$ then:
        \begin{equation*}
            m(b-a) \leq \int_a^b f(x) \dd x \leq M(b-a)
        \end{equation*}
    \end{enumerate}
    \item \textbf{Property 4 (First Mean Value Theorem):}
    \par Suppose $f(x)$ is integrable on $[a,b]$ and $m \leq f(x) \leq M \ \forall \ x \in [a,b]$, then there exists $\mu$ such that:
    \begin{equation*}
        \int_a^b f(x) \dd x = \mu (b-a), \qquad m< \mu < M
    \end{equation*}
    If $f(x)$ is continuous on $[a,b]$ then there exists $c \in [a,b]$ such that:
        \begin{equation*}
        \int_a^b f(x) \dd x = f(c) (b-a)
    \end{equation*}
    \item \textbf{Property 5 (Second Mean Value Theorem):}
    If we have these three conditions:
    \begin{enumerate}
        \item $f(x)$ and $f(x)g(x)$ are integrable on $[a,b]$
        \item $m \leq f(x) \leq M \ \forall \ x \in [a,b]$
        \item $g(x)$ does not change signs on $[a,b]$
    \end{enumerate}
    Then there exists $\mu$ such that:
    \begin{equation*}
        \int_a^b f(x)g(x) \dd x = \mu \int_a^b g(x) \dd x, \qquad m< \mu < M
    \end{equation*}
    If $f(x)$ is continuous on $[a,b]$ then there exists $c \in [a,b]$ such that:
    \begin{equation*}
        \int_a^b f(x)g(x) \dd x = f(c) \int_a^b g(x) \dd x
    \end{equation*}
\end{enumerate}
\subsection{Integral Functions}
Suppose $f(x)$ is an integrable function on $[a,b]$, then for all $x \in [a,b]$, $f$ is also integrable on $[a,x]$. We can then define the function $F(x) = \mathlarger{\int_a^x f(t) \dd t}$. We then have some very important foundational theorems:
\begin{concept}
    \begin{itemize}
        \item If $f(x)$ is integrable on $[a,b]$ then $F(x)$ is continuous on $[a,b]$
        \item If $f$ is continuous at $x_0 \in [a,b]$ then $F(x)$ is differentiable at $x_0$ and $$F'(x_0) = f(x_0)$$
        \item If $f(x)$ is continuous on the closed interval $[a,b]$ and $F(x)$ is an integral of $f(x)$ then:
        \begin{equation*}
            \int_a^b f(x) \dd x = F(b) - F(a)
        \end{equation*}
    \end{itemize}
\end{concept}
\subsection{Ways to Calculate Definite Integrals}
\begin{enumerate}
    \item \textbf{Integration by parts:}
    \par Suppose $u(x),v(x)$ are continuously differentiable functions on $[a,b]$. Then:
    \[
    \int_a^b u \, dv =  uv \Bigg|_a^b - \int_a^b v \, du
    \]
    \item \textbf{Change of variables:}
    \begin{enumerate}
        \item \textbf{Substitute $x = \varphi(t)$}
        \par Consider $I = \mathlarger{\int_a^b f(x) \dd x}$ with $f(x)$ being continuous on $[a,b]$. Substitute $x = \varphi(t)$ with the following three conditions:
        \begin{itemize}
            \item $\varphi(t)$ is has a continuous derivative on $[a,b]$
            \item $\varphi(a) = \alpha ; \ \varphi(b) = \beta$
            \item When $t$ changes from $\alpha$ to $\beta$ in $[\alpha,\beta]$ then $x = \varphi(t)$ continuously changes from $a$ to $b$
        \end{itemize}
        Then we have the following formula:
    \begin{equation*}
        \int_a^b f(x) \dd x = \int_{\alpha}^{\beta}f[\varphi(t)]\varphi'(t) \dd t
    \end{equation*}
    \item \textbf{Substitute $t = \varphi(x)$} 
    \par Suppose the integral we are trying to solve has the form $I = \mathlarger{\int_a^b f[\varphi(x)]\varphi'(x) \dd x}$, where $\varphi(x)$ is monotonic and is continuously differentiable on $[a,b]$. Then:
    \begin{equation*}
        \int_a^b f[\varphi(x)]\varphi'(x) \dd x = \int_{\varphi(a)}^{\varphi(b)}f(t) \dd t
    \end{equation*}
    \end{enumerate}
    \item \textbf{Recursion or Induction}
    \par We can look at an example. Calculate:
    \begin{equation*}
        I_n = \int_0^\frac{\pi}{2}\cos^n x \cos nx \dd x
    \end{equation*}
    We use integration by parts: Let $u = \cos^nx$ and $\dd v = \cos nx \dd x$. Then $v = \dfrac{1}{n}\sin nx$. Applying the integration by parts formula, we have:
    \begin{align*}
        I_n = & \dfrac{1}{n} \int_0^\frac{\pi}{2}\cos^n x \cos nx \dd x \\
        = & \dfrac{1}{n} \cos ^n x \sin nx \Big|_0^{\frac{\pi}{2}} + \dfrac{1}{n}  \int_0^{\frac{\pi}{2}} n \cos ^{n-1} x \sin x \sin nx  \dd x \\
        = &  \int_0^{\frac{\pi}{2}} \cos ^{n-1} x \sin x \sin nx  \dd x
    \end{align*}
    Then we see:
    \begin{align*}
        2I_n = & \int_0^\frac{\pi}{2}\cos^n x \cos nx \dd x + \int_0^{\frac{\pi}{2}} \cos ^{n-1} x \sin x \sin nx  \dd x \\
        = & \int_0^\frac{\pi}{2} \cos^{n-1}x (\cos x \cos nx + \sin x \sin nx) \dd x \\
        = & \int_0^\frac{\pi}{2} \cos^{n-1}x \cos(n-1)x \dd x \\
        =& I_{n-1}
    \end{align*}
    It's easy to compute $I_0 = \dfrac{\pi}{2}$. Then this is a geometric series, and $I_n = \dfrac{\pi}{2^{n+1}}$.
\end{enumerate}
\subsection{Important Results}
\begin{enumerate}
    \item \textbf{The Fundamental Theorem of Calculus}
    \begin{equation*}
        \dfrac{\dd y}{\dd x}\int_a^x f(t) \dd t = f(x)
    \end{equation*}
    And more generally:
    \begin{equation*}
        \dfrac{\dd y}{\dd x}\int_a^{g(x)} f(t) \dd t = f(g(x))g'(x)
    \end{equation*}  
    \item \textbf{Riemann Sum}
    \par Recall the formula from earlier:
    \begin{equation*}
        S_n = \sum_{i=0}^{n-1}f(\xi_i) \Delta x_i \qquad  \Delta x_i \in [x_i, x_{i+1}]
    \end{equation*}
    If we partition $[a,b]$ into $n$ subintervals with equal length using $a = x_0 < x_1 < \ldots < x_n = b$, where $x_i = a + (b-a)\dfrac{i}{n}$ then:
    \begin{equation*}
        S_n = \dfrac{b-a}{n} \sum_{i=0}^{n-1}f(\xi_i) \qquad  \xi_i \in [x_i, x_{i+1}]
    \end{equation*}
    If $f(x)$ is integrable on $[a,b]$ and choosing $\xi_i = x_i$, we have the left and right sums:
    \begin{important}
    \textbf{Choosing $\xi_i = x_i$ yields:}
    \begin{equation*}
        \lim_{n \rightarrow \infty} \dfrac{b-a}{n} \left[ \sum_{i=0}^{n-1}f\left(a+ \dfrac{b-a}{n}i \right)\right] = \int_a^b f(x) \dd x
    \end{equation*}
    \textbf{Choosing $\xi_i = x_{i+1}$ yields:}
    \begin{equation*}
        \lim_{n \rightarrow \infty} \dfrac{b-a}{n} \left[ \sum_{i=1}^{n}f\left(a+ \dfrac{b-a}{n}i \right)\right] = \int_a^b f(x) \dd x
    \end{equation*}
    \end{important}
    \item \textbf{Integral Equalities}
    \begin{itemize}
        \item $\mathlarger{\int_a^b f(x) \dd x = \int_a^b f(a+b-x) \dd x}$
        \item $\mathlarger{\int_0^{\frac{\pi}{2}}f(\sin x) \dd x = \int_0^{\frac{\pi}{2}}f(\cos x) \dd x}$
        \item $\mathlarger{\int_{-a}^a \dfrac{f(x)}{1+b^x}\dd x = \int_0^a f(x) \dd x}$
        \item $\mathlarger{\int_a^b x^m (a+b-x)^n \dd x  = \int_a^b x^n (a+b-x)^m \dd x}$
    \end{itemize}
    \item \textbf{Integral Inequalities}
    \begin{important}
        \textbf{Cauchy-Schwarz Inequality:}
        \begin{equation*}
            \left( \int_a^b f(x) g(x) \dd x \right)^2 \leq \left( \int_a^b f^2(x) \dd x \right)\left( \int_a^b g^2(x) \dd x \right)
        \end{equation*}
    \end{important}
\end{enumerate}
\section{Improper Integrals}
\subsection{Improper Integrals with infinite endpoints}
\par Suppose $f(x)$ is defined on the interval $[a,+\infty)$ and integrable on every closed intervals $[a,A]$, then:
\begin{concept}
    \textbf{Definition:} The limit of the integral $\mathlarger{\int_a^A f(x) \dd x}$ as $A \rightarrow +\infty$ is called an improper integral of the function $f(x)$ on $[a,+\infty)$ and is denoted by:
    \begin{equation*}
        \int_a^{+\infty} f(x) \dd x = \lim_{A \rightarrow \infty} \int_a^A f(x) \dd x
    \end{equation*}
\end{concept}
If this limit exists then we say the improper integral \emph{converges}. Conversely, if it does not exist or it is at infinity, we say it \emph{diverges}. Analogously, we can define improper integrals from negative infinity to some value, and on the entire real line $\mathbb{R}$.
\subsection{Improper Integrals of Unbounded functions}
\par Suppose $f(x)$ is defined on $[a,b)$ and integrable on all intervals $[a,t]$ such that $t<b$ and $\mathlarger{\lim_{x \rightarrow b} = \infty}$. The point $x=b$ is called a singularity point of the function $f(x)$.
\begin{concept}
    \textbf{Definition:} The limit of the integral $\mathlarger{\int_a^t} f(x) \dd x$ as $t \rightarrow b^-$ is called an improper integral of the function $f(x)$ on the interval $[a,b)$ and is denoted by:
    \begin{equation*}
        \int_a^b f(x) \dd x = \lim_{t \rightarrow b^-}\int_a^t f(x) \dd x
    \end{equation*}
\end{concept}
If this limit exists then we say the improper integral \emph{converges}. Conversely, if it does not exist or it is at infinity, we say it \emph{diverges}. Analogously, we can define improper integrals on $(a,b]$ and on $(a,b)$.
\par As for integrals with two critical points $x=a$ and $x=b$, we can split:
\begin{equation*}
    \int_a^b f(x) \dd x = \int_a^c f(x) \dd x + \int_c^b f(x) \dd x
\end{equation*}
\subsection{Convergence Criteria}
\textbf{Comparison criterium:}
\begin{enumerate}
    \item Given two functions $f(x)$ and $g(x)$ that are integrable on all finite intervals $[a,A]$ and:
    \begin{equation*}
        0 \leq f(x) \leq g(x)
    \end{equation*}
    Then:
    \begin{enumerate}
        \item If $\mathlarger{\int_a^{+ \infty}g(x) \dd x}$ converges then $\mathlarger{\int_a^{+ \infty}f(x) \dd x}$ converges.
        \item If $\mathlarger{\int_a^{+ \infty}f(x) \dd x}$ diverges then $\mathlarger{\int_a^{+ \infty}g(x) \dd x}$ diverges.
    \end{enumerate}
    \item Suppose $f(x)$ and $g(x)$ are functions integrable on every finite intervals $[a,A]$ and $\mathlarger{\lim_{x \rightarrow + \infty}} \dfrac{f(x)}{g(x)} = k>0$. Then the integrals $\mathlarger{\int_a^{+\infty}f(x) \dd x}$ and $\mathlarger{\int_a^{+\infty}g(x) \dd x}$ either both converges or both diverges.
\end{enumerate}
\textbf{Corollaries:}
\par Given $f(x)$ and $g(x)$ are two positive integrable functions on $[a,+\infty)$. Then:
\begin{enumerate}
    \item If $\mathlarger{\lim_{x \rightarrow + \infty} \dfrac{f(x)}{g(x)} = 0 }$ and $\mathlarger{\int_a^{+\infty}g(x) \dd x}$ converges then $\mathlarger{\int_a^{+\infty}f(x) \dd x}$ converges.
    \item If $\mathlarger{\lim_{x \rightarrow + \infty} \dfrac{f(x)}{g(x)} = +\infty }$ and $\mathlarger{\int_a^{+\infty}g(x) \dd x}$ diverges then $\mathlarger{\int_a^{+\infty}f(x) \dd x}$ diverges.
\end{enumerate}
Analogously we also have similar criteria for functions with singularities. Also there are two things to keep in mind:
\begin{important}
    \begin{enumerate}
        \item When determining the convergence of an improper integral, we only care about the rough behaviour of the function around abnormal points.
        \item When using the comparison criteria, we often use that in conjunction with the $p$-test:
        \begin{enumerate}
            \item \begin{equation*}
                \int_a^{+\infty} \dfrac{\dd x}{x^\alpha}
            \end{equation*}
            is convergent if $\alpha > 1$, and divergent if $\alpha \leq 1$.
            \item \begin{equation*}
                \int_a^b \dfrac{\dd x}{(x-a)^\alpha}
            \end{equation*}
            is convergent if $\alpha < 1$, and divergent if $\alpha \geq 1$.
            \item \begin{equation*}
                \int_a^b \dfrac{\dd x}{(b-x)^\alpha}
            \end{equation*}
            is convergent if $\alpha < 1$, and divergent if $\alpha \geq 1$.
        \end{enumerate}
    \end{enumerate}
\end{important}
\subsection{Convergence and Absolute Convergence}
\begin{equation*}
    \boxed{\int_0^1 \int_1^{\sqrt{y}}\dfrac{y}{x^5+1}\dd x \dd y}
\end{equation*}


\end{document}
